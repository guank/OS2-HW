\documentclass[letterpaper,10pt,draftclsnofoot,onecolumn]{IEEEtran}

\usepackage{graphicx}                                     
\usepackage{amssymb}                                         
\usepackage{amsmath}                                         
\usepackage{amsthm}                                          

\usepackage{alltt}                                           
\usepackage{float}
\usepackage{color}
\usepackage{url}
\usepackage{listings}

\usepackage{balance}
\usepackage[TABBOTCAP, tight]{subfigure}
\usepackage{enumitem}
\usepackage{pstricks, pst-node}

\usepackage{textcomp}
\usepackage[margin=0.75in]{geometry}

\parindent = 0.0 in
\parskip = 0.1 in


\begin{document}

\title{Processes, Threading and CPU Scheduling}

\author{
\IEEEauthorblockN{George Crary}
\IEEEauthorblockA{\\Operating Systems II\\Spring 2017}
}

\maketitle
\begin{abstract}
Between Windows, Linux and FreeBSD many similarities exist as in end as operating systems they fundamentally try to achieve the same things however differences do still exist. This analysis will inspect these details and the origins that led to them. While both FreeBSD and Linux share lineage from Unix (Linux via Minix), Windows does not obviously. As a result of this the differences between these operating systems can be interpreted by the level of developer control that they come from. For example the Linux operating system development is purely concerned with the Linux kernel itself while the FreeBSD organization involves itself with both kernel development and core system utilities. BSD in comparison to Linux is subject to more structured design as a result of this. Windows on other hand has development interests from the kernel all the way out to userland. Alternatively the historical environment that these operating systems were developed in also have an influence worth noting. FreeBSD’s Unix ancestors hail from the days of the PDP11 while FreeBSD and Linux themselves began for their lives on the x86 platform.  Windows NT at the very least on the other hand also began life targeting the x86 platform as well.
\end{abstract}
\pagebreak

\tableofcontents
\pagebreak
\section{Processes}
something. \cite{love}.
\par

\newpage
\bibliographystyle{IEEEtran}
\bibliography{ref}
\end{document}
