\documentclass[letterpaper,10pt,draftclsnofoot,onecolumn]{IEEEtran}

\usepackage{graphicx}                                     
\usepackage{amssymb}                                         
\usepackage{amsmath}                                         
\usepackage{amsthm}                                          

\usepackage{alltt}                                           
\usepackage{float}
\usepackage{color}
\usepackage{url}

\usepackage{balance}
\usepackage[TABBOTCAP, tight]{subfigure}
\usepackage{enumitem}
\usepackage{pstricks, pst-node}

\usepackage{textcomp}
\usepackage[margin=0.75in]{geometry}

\parindent = 0.0 in
\parskip = 0.1 in

\begin{document}

\title{Project 1: Getting Acquainted}

\author{
\IEEEauthorblockN{Kevin Guan}
\IEEEauthorblockN{George Crary}
\IEEEauthorblockA{\\Operating Systems II\\Group 13-08\\Spring 2017}
}

\maketitle
\begin{abstract}
This paper discusses the processes that were practiced in the first project for the CS444
class. The topics included the Linux Kernel and the learning of parallelism within an operating 
system. The Linux Kernel was explored via learning how to configure Qemu. For parallelism, we
learned the fundamentals of threading to solve a given concurrency problem. 
\end{abstract}
\pagebreak

\section{Command Log for Linux Kernel and Qemu}
\subsection{Terminal One}
\begin{enumerate}
\item cd /scratch/spring2017
\item mkdir 13-08
\item cd 13-08
\item git clone git://git.yoctoproject.org/linux-yocto-3.14
\item cd linux-yocto- 3.14
\item git checkout v3.14.26
\item source /scratch/opt/environment-setup-i586-poky-linux.csh
\item cp /scratch/spring2017/files/config-3.14.26-yocto-qemu .config
\item make menuconfig
\item make -j4 all
\item cd ..
\item gdb
\item target remote :6808
\item continue
\end{enumerate}

\subsection{Terminal Two}
\begin{enumerate}
\item cp /scratch/spring2017/files/bzImage-qemux86.bin .
\item cp /scratch/spring2017/files/core-image- lsb-sdk-qemux86.ext3 .
\item qemu-system- i386 -gdb tcp::6808 -S -nographic -kernel bzImage-qemux86.bin -drive file=core-image- lsb-sdk- qemux86.ext3,if=virtio -enable-kvm -net none -usb -localtime -- no-reboot -- append "root=/dev/vda rw console=ttyS0 debug"
\item qemu-system- i386 -gdb tcp::6808 -S -nographic -kernel linux-yocto- 3.14/arch/x86/boot/bzImage  -drive file=core-image- lsb-sdk- qemux86.ext3,if=virtio -enable- kvm -net none -usb -localtime -- no-reboot -- append "root=/dev/vda rw console=ttyS0 debug"
\end{enumerate}

\section{Explanation of the Flags in the Qemu Command-Line}


\section{Concurrency: The Producer-Consumer Problem}
\subsection{What do you think the main point of this assignment is?}
The main purpose of this assignment is to familiarize ourselves and build a stronger understanding with 
the concepts of parallelism via the application of pthreads. It also brings to light the importance of the 
necessary procedures, like locks, that must take place during multi-process operations.
\subsection{How did you personally approach the problem? Design decisions, algorithm, etc.}
In the beginning of process, the first priority was pinpointing and noting the restraints and expectations for the problem.
This process required reading the instructions numerous times and drawing several flow charts of the different functionalities that will
lead to the expected output. From there, the different parts of the problem were compartmentalized and focused individually on seperate time blocks.
The initial part of the assignment involved setting up the structs with the parameters that were given on the instructions. Afterwards,
the process became trickier as pthreads, rdrand, and Marsenne Twister were foreign territory. As a result, the topics were researched thoroughly
and made progress as we slowly grew more familiar with the concepts alongside much trial and error. 
\subsection{How did you ensure your solution was correct? Testing details, for instance.}
For testing, the buffer was decreased to very small value in order to better observe its functionalities at a micro scale. This practice allowed for 
better observation of the blocking that needs to take place if the buffer is empty or full. To better understand of the processes that take place,
print statements were also placed to better identify when specific actions were done or not done as intended.
\subsection{What did you learn?}
One of the lessons from completing the concurrency is a stronger understanding of working with paralleism via applying pthreads. Another 
interesting aspect that was learned was the application of rdrand and the potential of its powerful randomization application. 

\section{Version Control Log}
\begin{tabular}{|p{0.3\linewidth}|p{0.3\linewidth}|}
\hline
\textbf{Date}&\textbf{Changes}\\
\hline
4/12/16 & Started assignment\\
\hline

\end{tabular}

\section{Work Log}
\begin{tabular}{|p{0.3\linewidth}|p{0.3\linewidth}|p{0.3\linewidth}|}
\hline
\textbf{Date}&\textbf{Task}&\textbf{Detail}\\
\hline
4/12/16 & Started Stuyding the Threads & Got there\\
\hline

\end{tabular}

\end{document}
